\documentclass[11pt,twoside]{article}
\usepackage[T1]{fontenc}
\usepackage[english]{babel}
\usepackage[utf8]{inputenc}
%~ \usepackage{amsmath}
\usepackage{amscd}
\usepackage{amssymb}
\usepackage{multirow}
\usepackage{tabularx}
\usepackage{url}
\usepackage{fancyhdr}
\usepackage{lastpage}
\usepackage{mathtools}
\usepackage[a4paper,margin=2.5cm,hmarginratio=1:1]{geometry}
\usepackage{graphicx}

\title{Kometens Stoftsvans}
\author{Joakim Uddholm, juddholm@kth.se, 9110013290 \\
		Jo Tryti, tryti@kth.se, 8612050438}
\date{}

\begin{document}
\maketitle

\section{Problembeskrivning}


\section{Kometens bana}

\section{Kometens riktning}	
Med radien $r(\phi)$ fås en positionsvektor $P(\phi)$.
\begin{equation}
	P(\phi) = \begin{cases}
	x = r(\phi) cos(\phi) \\
	y = r(\phi) sin(\phi)
	\end{cases}
\end{equation}
Deriveras ovan funktion ges riktningen $k$ som en funktion av $\phi$. 
\begin{equation}
	k = \begin{cases}
	x = r'(\phi) cos(\phi) - r(\phi) sin(\phi) \\
	y = r'(\phi) sin(\phi) + r(\phi) cos(\phi)
	\end{cases}
\end{equation}

\section{Randvärdet kometens hastighet}



\section{Stoftpartiklarna}



\end{document}
